%\chapter*{Summary}
%\markright{Summary}

%Summary \ldots

\chapter*{Abstract}
\markright{Abstract}

The aim of this thesis is to implement an application which extracts publicly available data about the Austrian parliament, performs analysis on this data and visualizes the results. As information source, politician profiles and the transcripts of the sessions of the national council are being used. These semi-structured documents get collected, are being transformed into a structured form and get persisted into a database. The extracted data includes politicians and their mandates, sessions of the national council and the speeches within these sessions. A majority of these speeches is tagged with pro or con. This enables the calculation of relation weights between politicians and parliamentary clubs, respectively. Furthermore, relation graphs can be created which show by visual cues how similar the attitudes of different politicians/clubs are. Other analysis includes inner government/opposition cohesion, government-opposition relation, absence measures and a few other simple measures. All the taken analysis measures also get visualized within a web application to represent the information in a more user friendly way.