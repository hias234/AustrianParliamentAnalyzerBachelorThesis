\documentclass[12pt]{report}
\usepackage[utf8]{inputenc}
\usepackage{graphicx}
\graphicspath{ {images/} }

\title{Austrian Parliament Analyzer}
\author{Markus Hiesmair}

\begin{document}

\pagenumbering{gobble}
\maketitle
\newpage

\chapter*{Abstract}
Abstract goes here

\tableofcontents

\pagenumbering{arabic}

\chapter{Introduction}
One of the most crucial requirements of a democracy is transparency. There are several ways how one can gain information about the current and past political activities in Austria. One of them are the publicly available protocols of the national council sessions. In these protocols every word said is being written down. Therefore, they are very long and it is hard to gain meaning out of it.

To be able to analyze and visualize the activities and relations of the politicians and political parties in a better way, during this thesis the protocols are being extracted, analyzed and visualized. 

\section{Research Goals}
The protocols are currently available in semi-structured form - through HTML-files. To be able to properly persist and analyze the data the protocols should be transformed into a fully structured form (e.g. Java Objects). The following data will be extracted:

\begin{itemize}
  \item Legislative periods and its sessions
  \item Politicians and their mandates
  \item Parliament clubs
  \item Discussions and speeches during the sessions
\end{itemize}

As soon as this is done, the extracted data should be persisted into a arbitrary relational database. Furthermore, some general and network analysis should be done on the data. The following list includes some of the analysis.

\begin{itemize}
  \item Find groups of politicians (or parliament clubs) with the same attitudes.
  \item Analyze how homogeneous the attitudes of politicians of the same parliament club are.
  \item Find the politicians which take part in the most discussions.
  \item Find the most absent national council members.
\end{itemize}

In the final step the results of the analysis should be visualized via a web application. The focus hereby lies in making the results as easy to understand as possible.

\section{Political System in Austria}
Description of the political System.

\chapter{Related Work}
Related Work.

\chapter{Design and Implementation}

Overview... Architecture... General Components

\section{Data Extraction and Transforming}

\section{Export into Database}

\section{Analysis}

\section{Visualization}

\chapter{Results}
Results

\section{Relations of Parliament Clubs}
Graph + Explanations

\section{Relations of Politicians}
Graph + Explanations

\chapter{Conclusion}
Conclusion

\bibliography{APA_Bakk_Thesis} 
\bibliographystyle{ieeetr}

\end{document}
