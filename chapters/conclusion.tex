\externaldocument{eval}

\chapter{Conclusions and Future Work}
\label{chap:conclusion}

The implemented prototype shows that it is possible to get structured data out of the stenographic transcripts in good quality and that sophisticated analysis can be performed on the basis of this information. The applied analysis can give an overview of politician absence, length of serving in the national council, politician activity, the relation between politicians/clubs and the inner cohesion of coalition or opposition. Especially the analysis of the relationships among parliamentary clubs/politicians brought interesting results. The resulting relation graphs showed that in the national council of the Austrian parliament, there are always two groups of politicians/clubs: Government and opposition. This fact does not depend on the clubs which are in coalition. If two clubs were together in coalition and had a positive relationship in one period, and in the next period one is in coalition whereas the other is in opposition, the two parties will then have a negative relationship. The analysis of the overall relation between government and opposition confirmed this fact (see section \ref{sec:gov_opp_relation}).

While the results of the analysis visualized interesting facts, there is still a lot work which can be done. With the existing data, politicians can be clustered to find groups and subgroups in the national council automatically in arbitrary size, similar to the work of Porter and Newman \cite{Porter_2005}. Furthermore, sentiment analysis could be applied on the texts of the speeches in the parliament to find out whether a speech was positive or negative. This could help to improve the calculation of the relation weights between politicians and therefore also result in better graphs. Another potential future work can be the extraction of data from the federal council and other chambers of the political system in Austria and to combine them with the current data to gain connections between the different chambers and to be able to discover political structures in a broader context.