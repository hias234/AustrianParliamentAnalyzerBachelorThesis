\chapter{Introduction}
\label{sec:introduction}

One of the most crucial requirements of a democracy is transparency. There are several ways how one can gain information about the current and past political activities in Austria. One of the best possibilities among them are the publicly available protocols of the national council sessions. In these protocols every word said in a session is written down and that makes up the corresponding protocol. Unfortunately, these protocols are very long and it is hard to gain meaning out of it, because of its plain and simple structure and the great amount of data.

To be able to analyze and visualize the activities and relations of the politicians and parties in a better way, during this thesis the protocols are being extracted, transformed, analyzed and visualized.

\section{Research Goals}
The protocols are currently available in semi-structured form - through HTML files. To be able to properly persist and analyze the data, the protocols have to be transformed into a fully structured form (e.g. Java Objects). The following elements will be extracted:
\begin{itemize}
  \item Legislative periods and their sessions
  \item Politicians and their mandates
  \item Parliament clubs
  \item Discussions and speeches during the sessions
  \item Relations among the above listed elements
\end{itemize}

As soon as this is done, the extracted data should be persisted into a arbitrary relational database. Furthermore, some general and network analysis should be done on the data. The following list includes some of the analysis.
\begin{itemize}
  \item Find groups of politicians (or parliament clubs) with the same attitudes.
  \item Analyze how homogeneous the attitudes of politicians of the same parliament club are.
  \item Find the politicians which take part in the most discussions.
  \item Find the most absent national council members.
\end{itemize}

In the final step the results should be visualized via a web application. The focus hereby lies in making the results as easy to understand as possible.

\section{Political System in Austria}
To be able to understand all the terms of the Austrian political system and to understand the general process of the law making, in this section there will be a short introduction into the field.

In Austria there is an indirect democratic system. This means that the Austrian people elect politicians (or in a more abstract way the political parties) which then represent them and try to speak for them.

\subsection{Elections}
In a national scope the most important institution is the parliament. Every five years - this is the length of a legislative period - there are federal elections. In these elections the Austrian people elect parties and depending on the results these parties get a specific number of places in the national council, called mandates. For each mandate, a party is allowed to send out a politician which can participate in national council. Altogether, there are 183 politicians in the Austrian parliament.

