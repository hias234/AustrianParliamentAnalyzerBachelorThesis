\chapter{Introduction}
\label{sec:introduction}
One of the most crucial requirements of a democracy is transparency. There are several ways how one can gain information about the current and past political activities in Austria. One of the best possibilities among them are the publicly available protocols of the national council sessions. In these protocols every word said in a session is written down and that makes up the corresponding protocol. Unfortunately, these protocols are very long and it is hard to gain meaning out of it, because of its plain and simple structure and the great amount of data.

To be able to analyze and visualize the activities and relations of the politicians and parties in a better way, during this thesis the protocols are being extracted, transformed, analyzed and visualized.

\section{Research Goals}
The protocols are currently available in semi-structured form - through HTML files. To be able to properly persist and analyze the data, the protocols have to be transformed into a fully structured form (e.g. Java Objects). The following elements will be extracted:
\begin{itemize}
  \item Legislative periods and their sessions
  \item Politicians and their mandates
  \item Parliament clubs
  \item Discussions and speeches during the sessions
\end{itemize}

As soon as this is done, the extracted data should be persisted into a arbitrary relational database. Furthermore, some general and network analysis should be done on the data. The following list includes some of the analysis.
\begin{itemize}
  \item Find groups of politicians (or parliament clubs) with the same attitudes.
  \item Analyze how homogeneous the attitudes of politicians of the same parliament club are.
  \item Find the politicians which take part in the most discussions.
  \item Find the most absent national council members.
\end{itemize}

In the final step the results should be visualized via a web application. The focus hereby lies in making the results as easy to understand as possible.

\section{Austrian Parliament}
In this section there will be a short overview over the activities and tasks of the Austrian parliament, so that in the following sections the basic process will be clear.

The parliament basically consists of two chambers, the national council and the federal council. The national council is elected through federal elections, whereas the federal council consists of delegates of the 9 provinces. Both chambers have different functions and their goal is to ensure that the decisions are in the best interest for the Austrian people.

\subsection{National Council}
The national council consists of 183 members, which can band together to form so called parliamentary clubs. Usually there is a club per party, which is in the national council.

The tasks of the national council include law-making, controlling the government, seeking solutions, determining the budget and much more. 

\subsection{Federal Council}
