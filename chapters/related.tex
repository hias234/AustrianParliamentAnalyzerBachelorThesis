\chapter{Related Work}
\label{chap:relatedwork}

In the context of computer science, there are only a few works on automatically analyzing political structures such as a parliament. In 2013, Renzo Lucioni \cite{Lucioni_2015} used publicly available voting data from the Congress of the United States of America to analyze the relationships among politicians and how distinct the two main parties are. To achieve this, he used data from the $101^{st}$ Congress through the $113^{th}$ Congress and created network graphs which graphically showed which politicians vote similar. He also showed how the structure of the Congress developed over time. His results showed that the gap between the Republicans and the Democrats - the only two really relevant parties in the USA - became larger and larger over the last decades. This means that both parties vote more and more against each other. In the context of the Austrian parliament, similar analysis can be applied, if data is available in sufficient quality. For example, it can be analyzed which parties vote similar and if there exist relations between parties which are in the government and in the opposition. You can find the results for the Austrian parliament in Section \ref{sec:relations_clubs}.

An earlier work was done by Porter and Newman in 2005 \cite{Porter_2005}. They wrote a paper on network analysis of committees in the U.S. House of Representatives and tried to show the connections between representatives of the House and the committees and subcommittees. In these committees happens a big amount of the American legislation and especially the assignment of politicians and the change of it over time are interesting subjects of analysis. In their work Porter and Newman gain information without specific knowledge of the structure of the committees, using technologies of network analysis. In particular, they tried to find communities and their connections within the network of the committees to get information about strategic assignment of politicians in important committees. Furthermore, Porter and Newman used single-linkage clustering to get clusters of communities and their connections and also visualized that with a dendrogram representing the hierarchical structure of the committees.

Using techniques of network analysis they found out that there are certain correlations between communities and that there also high relations between a political position and the assignment to a committee. 

In 2012, Amelio \cite{Amelio_2012} also did a study on the voting behavior in Italy. She used multidimensional scaling, hierarchical clustering and network analysis.






