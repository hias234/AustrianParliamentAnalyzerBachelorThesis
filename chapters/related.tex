\chapter{Related Work}
\label{chap:relatedwork}

In the context of computer science, there are only a few works on automatically analyzing political structures such as a parliament. In 2013, Renzo Lucioni \cite{Lucioni_2015} used voting data from the Congress of the United States of America to analyze the relationships among politicians and how distinct the two main parties in the USA are. Therefore, he used data from the 101th Congress through the 113th Congress to also show how the structure of the Congress developed over time. His results show that the gap between the Republicans and the Democrats - the only two really relevant parties in the USA - becomes larger and larger. This means that both parties vote more and more against each other.

One among them is a paper on network analysis of committees in the U.S. House of Representatives \cite{Porter_2005}. Newman et al. and his colleagues tried to show the connections between representatives of the House and the committees and subcommittees, in which a big amount of the American legislation happens. Using techniques of network analysis they found out that there are certain correlations between communities and that there also high relations between a political position and the assignment to a committee. 

In 2012, Amelio \cite{Amelio_2012} also did a study on the voting behavior in Italy. She used multidimensional scaling, hierarchical clustering and network analysis.






