\chapter{Related Work}
\label{chap:relatedwork}

In the context of computer science, there are only a few works on automatically analyzing political structures such as a parliament. In 2013, Renzo Lucioni \cite{Lucioni_2015} used publicly available voting data from the Congress of the United States of America to analyze the relationships among politicians and how distinct the two main parties, the Democrats and the Republicans, are. To achieve this, he used data from the $101^{st}$ Congress through the $113^{th}$ Congress and created network graphs which graphically showed which politicians vote similar. He also showed how the structure of the Congress developed over time by creating several graphs of the years 1989 to 2013. His results showed that the gap between the Republicans and the Democrats became larger and larger over the last decades. This means that both parties vote more and more against each other. In the context of the Austrian parliament, similar analysis can be applied, if data is available in sufficient quality. For example, it can be analyzed which parties vote similar and if there exist relations between parties which are in the coalition and parties which are in the opposition. You can find the results for the Austrian parliament in section \ref{sec:relations_clubs}.

An earlier work was done by Porter and Newman in 2005 \cite{Porter_2005}. They wrote a paper on network analysis of committees in the U.S. House of Representatives and tried to show the connections between representatives of the House and the committees and subcommittees. There happens a big amount of the American legislation in these committees and especially the assignment of politicians and the change of it over time are interesting subjects of analysis. In their work Porter and Newman gain information without specific knowledge of the structure of the committees, using technologies of network analysis. In particular, they tried to find communities and their connections within the network of the committees to get information about strategic assignment of politicians in important committees. Furthermore, Porter and Newman used single-linkage clustering to get clusters of communities and their connections and also visualized that with a dendrogram representing the hierarchical structure of the committees and subcommittees. Similar analysis would also be interesting for the Austrian parliament, but community detection and clustering are not included within the scope of this thesis. In the second part of Porter and Newman's paper, they also have done some analysis on the relations among politicians in the House of Representatives. The results show the most left, most right and most partisan politicians in the House. This is especially interesting because it shows that all the most left politicians are Democrats and all most right politicians are Republicans, which shows that also the House of Representatives is divided at a high degree in Democrats and Republicans.

In 2012, Amelio \cite{Amelio_2012} also did a study on the voting behavior in the Italian parliament. One part of her study was analyzing party cohesion (how homogeneous all politicians of a specific party voted in the selected periods). An interesting result was that the cohesion of the parties in the opposition increased over time whereas the cohesion of the governing parties decreased and after the analyzed period the government was not reelected. Another measure taken was the parliamentarian similarity. This measure compares the voting behaviors of two parliamentarians and gives a result on how similar they voted. Based on the values obtained, Amelio did hierarchical clustering using single-linked clustering to find communities within the parliament and visualized the results in a dendrogram, similar to the result of Porter and Newman \cite{Porter_2005}. 

All three papers, which were discussed, show that through automatic analysis of political structures, information on the structure and clustering of political systems can be gained. Furthermore, through visualizations in graphs the information can be presented in a way everybody understands it easily and therefore the visualizations can be used to improve the general understanding of political systems and the current structures of parties and politicians.

\section{Related Work in the context of the Austrian Parliament}
Austria is no country with good information laws in terms of governmental transparency. In the last five years (from 2011 to 2015) in the "Right To Information"-Rating Austria was on the last place each time and in 2015 there were 102 nations investigated on their governmental transparency laws \cite{Informationsfreiheit_2015}!

To improve the transparency of political and governmental processes in Austria, in 2010 an organization with the title "Informationsfreiheit" (Information freedom) \cite{Informationsfreiheit_2015} was founded. They held several online campaigns and fought in several law suits for the improvement of the right to information and already had quite good success. Furthermore, they started a project called "OffenesParlament.at" (open parliament) \cite{OffenesParlament_2015} which has the goal to give a better overview of the data which is available on the Austrian parliament website \cite{AustrianParliament_2015}. Therefore, they also extract the required information out of the openly available HTML documents used in the context of this thesis. On their website, there will be a lot of documents of the parliamentary process available and it will presented and a more user friendly way. A few of the items presented are: discussions grouped by topic, speeches of politicians, bills and bill drafts and statements of politicians to bill drafts. As you can see the data, which OffenesParlament.at extracts is mostly the same which is extracted within this thesis.

